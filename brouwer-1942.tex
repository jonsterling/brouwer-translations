\documentclass{amsart}
\usepackage{brouwer}

\newcommand\Rank[1]{\varrho\left(#1\right)}
\newcommand\Max[1]{\zeta\left(#1\right)}
\begin{document}

\Title{Mathematics}{%
  Demonstration that the concept of spreads of higher order does not come into
  consideration as a fundemental notion in intuitionistic mathematics.
}{%
  Prof.\ \textsc{L.\ E.\ J.\ Brouwer}\footnoteEd{Translated from the original German by Jon Sterling.}
}

\Communicated{%
  Communicated at the meeting of September 26, 1942.%
}

In my note, \emph{``Concerning the free development of spreads and
functions''},\footnoteT{Proc.\ Ned.\ Akad.\ v.\ Wetensch. Amsterdam,
\textbf{45}, 322 (1942).} the process $M_\sigma$ was considered, through which
the fundamental sequence $\digamma'$, which is enumerated in an arbitrary,
predetermined way, is associated one-to-one with finite choice sequences of
numbers and likewise an arbitrary element $\sigma$ of the spread\footnoteT{For
the sake of simplicity, we restrict ourselves in this note to such spreads, in
the process of whose creation neither inhibition nor termination occurs. This
restriction is inessential.} $M$. We want to call this process $M_\sigma$ a
spread \emph{of second order}, and the successions of figure-sequences thus
associated to the unrestricted choice sequences of numbers \EdMark{[we shall
call]} \emph{the elements of the second-order spread $M_\sigma$}.

The assertion stated in my quoted note, that $M_\sigma$ acts as a subspecies of
a spread $M_1$ which is derivable from $M$, and that the union of all
$M_\sigma$ generated from $M$ is identical with this $M_1$, shall be
demonstrated as follows. First, we will deal with the construction of the
spread $M_1$.

Let $\alpha_1\, \alpha_2\, \dots\, \alpha_m$ be a finite choice sequences of
numbers. We will indicate the rank thereof in the fundamental sequence
$\digamma'$ with $\Rank{\alpha_1\,\alpha_2\,\dots\,\alpha_m}$ and the maximum
of the numbers $\Rank{\alpha_1},\ \Rank{\alpha_1\,\alpha_2},\ \dots\
\Rank{\alpha_1\,\alpha_2\,\dots\,\alpha_m}$ with
$\Max{\alpha_1\,\alpha_2\,\dots\,\alpha_m}$.

We will call the combination of an arbitrary number $\alpha_1$ with
$\Rank{\alpha_1}$ arbitrary numbers
$\beta_1\,\beta_2\,\dots\,\beta_{\Rank{\alpha_1}}$ a $K$-combination. We
enumerate the $K$-combinations through a fundamental sequence $\digamma$. We
indicate any $K$-combination which receives the rank $\nu_1$ in $\digamma$ with
$K_{\nu_1}$.

\end{document}

