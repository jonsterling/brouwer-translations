\documentclass{amsart}
\usepackage{brouwer}

\newcommand\Rank[1]{\varrho\left(#1\right)}
\newcommand\Max[1]{\zeta\left(#1\right)}

\begin{document}

\Title{Mathematics}{%
  Demonstration that the concept of spreads of higher order does not come into
  consideration as a fundemental notion in intuitionistic mathematics.
}{%
  Prof.\ \textsc{L.\ E.\ J.\ Brouwer}\footnoteEd{Translated from the original German by Jon Sterling.}
}

\Communicated{%
  Communicated at the meeting of September 26, 1942.%
}

In my note, \emph{``Towards the free development of spreads and
functions''},\footnoteT{Proc.\ Ned.\ Akad.\ v.\ Wetensch. Amsterdam,
\textbf{45}, 322 (1942).} the process $M_\sigma$ was considered, through which
the fundamental sequence $\digamma'$, which is enumerated in an arbitrary,
predetermined way, is associated one-to-one with finite choice sequences of
numbers and likewise an arbitrary element $\sigma$ of the spread\footnoteT{For
the sake of simplicity, we restrict ourselves in this note to such spreads, in
the process of whose creation neither inhibition nor termination occurs. This
restriction is inessential.} $M$. We want to call this process $M_\sigma$ a
spread \emph{of second order}, and the successive sequences of figures thus
associated to the unrestricted choice sequences of numbers \EdMark{[we shall
call]} \emph{the elements of the second-order spread $M_\sigma$}.

The assertion stated in my quoted note, that $M_\sigma$ acts as a subspecies of
a spread $M_1$ which is derivable from $M$, and that the union of all
$M_\sigma$ generated from $M$ is identical with this $M_1$, shall be
demonstrated as follows. First, we will deal with the construction of the
spread $M_1$.

Let $\Sequence{\alpha_1, \alpha_2, \dots, \alpha_m}$ be a finite choice
sequences of numbers. We will notate the rank thereof in the fundamental
sequence $\digamma'$ with $\Rank{\Sequence{\alpha_1,\alpha_2,\dots,\alpha_m}}$
and the maximum of the numbers $\Rank{\alpha_1},\
\Rank{\Sequence{\alpha_1,\alpha_2}},\ \dots\
\Rank{\Sequence{\alpha_1,\alpha_2,\dots,\alpha_m}}$ with
$\Max{\Sequence{\alpha_1,\alpha_2,\dots,\alpha_m}}$.

We will call the combination of an arbitrary number $\alpha_1$ with
$\Rank{\alpha_1}$ arbitrary numbers
$\Sequence[,\,]{\beta_1,\beta_2,\dots,\beta_{\Rank{\alpha_1}}}$ a $K$-combination. We
enumerate the $K$-combinations through a fundamental sequence $\digamma$. We
notate any $K$-combination which receives the rank $\nu_1$ in $\digamma$ with
$K_{\nu_1}$.

For a given $\nu_1$, and thence also a given $\alpha_1$, and arbitrary
$\alpha_2$, we call the number $\alpha_2$ a $K_{\nu_1}$-combination in case
$\Max{\Sequence{\alpha_1,\alpha_2}} = \Max{\alpha_1}$, and \EdMark{likewise},
in case $\Max{\Sequence{\alpha_1,\alpha_2}} > \Max{\alpha_1}$, we call the
combination of $\alpha_2$ with $\Max{\Sequence{\alpha_1,\alpha_2}} -
\Max{\alpha_1}$ arbitrary numbers
$\beta_{\Max{\alpha_1}+1},\,\dots\,\beta_{\Max{\Sequence{\alpha_1,\alpha_2}}}$
\EdMark{a $K_{\nu_1}$-combination}. For each $\nu_1$ we enumerate the
$K_{\nu_1}$-combinations through a fundamental sequence $\digamma_{\nu_1}$. We
notate each $K_{\nu_1}$-combination, which receives rank $\nu_2$ in
$\digamma_{\nu_1}$, with $K_{\Sequence{\nu_1,\nu_2}}$.

For arbitrary $\nu_1$ and $\nu_2$, and thence also given $\alpha_1$ and
$\alpha_2$, and arbitrary $\alpha_3$, we call the number $\alpha_3$ a
$K_{\Sequence{\nu_1,\nu_2}}$-combination in case $\Max{\Sequence{\alpha_1,\alpha_2,\alpha_3}} =
\Max{\Sequence{\alpha_1,\alpha_2}}$, and, in case $\Max{\Sequence{\alpha_1,\alpha_2,\alpha_3}} >
\Max{\Sequence{\alpha_1,\alpha_2}}$, we \EdMark{likewise} call the combination of
$\alpha_3$ with $\Max{\Sequence{\alpha_1,\alpha_2,\alpha_3}} - \Max{\Sequence{\alpha_1,\alpha_2}}$
arbitrary numbers $\beta_{\Max{\Sequence{\alpha_1,\alpha_2}} +
1},\,\dots\,\beta_{\Max{\Sequence{\alpha_1,\alpha_2,\alpha_3}}}$ \EdMark{a
$K_{\Sequence{\nu_1,\nu_2}}$-combination}. For each pair of numbers $\nu_1,
\nu_2$ we enumerate the $K_{\Sequence{\nu_1,\nu_2}}$-combinations through a
fundamental sequence $\digamma_{\Sequence{\nu_1,\nu_2}}$. We notate each
$K_{\Sequence{\nu_1,\nu_2}}$ combination which receives the rank $\nu_3$ in
$\digamma_{\Sequence{\nu_1,\nu_2}}$ with $K_{\Sequence{\nu_1,\nu_2,\nu_3}}$.

Proceeding in this way, we define $K_{\Sequence{\nu_1,\nu_2,\dots,\nu_s}}$ for each natural
number $s$. In doing so, we take care from the outset to determine a
law through which the fundamental sequences $\digamma_{\Sequence{\nu_1,\nu_2,\dots,\nu_s}}$,
each enumerating $K_{\Sequence{\nu_1,\nu_2,\dots,\nu_s}}$, shall be defined once and for all.

The construction of $M_1$ will now be now be carried out, in which we shall
associate each sequence of figures with the finite sequence
$\Sequence{\nu_1,\nu_2,\dots,\nu_s}$, which is associated to the finite choice sequence
$\Sequence{\beta_1,\beta_2,\dots,\beta_{\Rank{\Sequence{\alpha_1,\alpha_2,\dots,\alpha_s}}}}$
for the respective numbers $\Sequence{\alpha_1,\alpha_2,\dots,\alpha_s},\
\Sequence{\beta_1,\beta_2,\dots,\beta_{\Rank{\Sequence{\alpha_1,\alpha_2,\dots,\alpha_s}}}}$
in $M$.

\bigskip

Let $\sigma$ be the element of $M$ generated by the infinite choice sequence
$\Sequence{\gamma_1,\gamma_2,\gamma_3,\dots}$. \emph{Then, the second-order
spread $M_\sigma$ is identical with a subspecies $\leftidx{_\sigma}{M}{_1}$ of
$M_1$.} This $\leftidx{_\sigma}{M}{_1}$ arises when for each $s$ in $M_1$ only
such $\nu_s$ may be chosen to which
$K_{\Sequence{\nu_1,\nu_2,\dots,\nu_{s-1}}}$-combinations correspond, in which
each $\beta_\tau$ is equal to the $\gamma_\tau$ which carries the same
index.

Conversely, let $e$ be an arbitrary element of $M_1$. Then the unrestricted
choice sequence $\Sequence{\nu_1,\nu_2,\nu_3,\dots}$ which generates $e$ in
$M_1$ is defined following the aforementioned definition of $\nu_s$
simultaneously with an unrestricted sequence of numbers
$\Sequence{\beta_1,\beta_2,\beta_3,\dots}$, which in turn generates the element
$\sigma(e)$ in $M$. The associated $M_{\sigma(e)}$ receives $e$ as an element.
\emph{Consequently, $M_1$ is identical to the union of all $M_\sigma$ generated
from $M$.}

From the above, it follows that the concept of spreads of second order does not
come into consideration as a fundamental notion in intuitionistic mathematics.

\bigskip

In order to define the concept of \emph{spreads of higher order}, we accept
that the concept of \emph{spreads of $n$th order} may be defined already, and
consider the process $\left(M^{(n)}\right)_\sigma$, through which the
fundamental sequence $\digamma'$, which is enumerated in an arbitrary and
predetermined way, will be associated one-to-one with finite choice sequences
and likewise an element $\sigma$ an element $\sigma$ of the $n$th-order spread
$M^{(n)}$. We will call this process $\left(M^{(n)}\right)_\sigma$ an
\emph{$(n+1)$th-order spread}, and we will call the successive sequences of
figures which are thus associated to the unrestricted choice sequences of
numbers \emph{the elements of the $(n+1)$th-order spread
$\left(M^{(n)}\right)_\sigma$}.

However, now a spread of second order $M_\sigma$ is a subspecies of a spread
$M$ which takes $M_\sigma$ as its basis, and is derivable from the spread
$M_1$, i.e. an arbitrary element $\pi$ of $M\sigma$ is simultaneously an
element of $M_1$. Hence, it follows that the third-order spread
$\left(M_\sigma\right)_\pi$ is identical with the second-order spread
$\left(M_1\right)_\pi$, i.e. an arbitrary third-order spread is identical with
a second-order spread. And hence it follows moreover that also for arbitrary
$n$ an arbitrary $n$th-order spread is identical with a second-order spread.

Consequently, it emerges that also the concept of higher-order spreads, in
contrast to the concept of higher-order species, does not come into
consideration as a fundamental notion in intuitionistic mathematics.

\end{document}

